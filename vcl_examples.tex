% chapter included in vclmanual.tex
\documentclass[vcl_manual.tex]{subfiles}
\begin{document}


\chapter{Examples}\label{chap:Examples}
\flushleft

This example calculates the polynomial $x^3 + 2\cdot x^2 - 5\cdot x + 1$ on a floating point vector. The order of calculation is specified by parentheses in order to make shorter dependency chains.

\begin{example}
\label{examplePolynomial}
\end{example}
\begin{lstlisting}[frame=single]
Vec4f polynomial (Vec4f x) {
    return (x + 2.0f) * (x * x) + ((-5.0f) * x + 1.0f);
}
\end{lstlisting}
\vspacebig

In 64-bit Windows, you may add \codei{\_\_vectorcall} and use a Clang or Microsoft compiler. This makes sure that vector parameters are transferred in registers rather than in memory. This is not needed when the function is inlined or when compiling for other platforms than Windows:

\begin{example}
\label{examplePolynomialVectorcall}
\end{example}
\begin{lstlisting}[frame=single]
Vec4f __vectorcall polynomial (Vec4f x) {
    return (x + 2.0f) * (x * x) + ((-5.0f) * x + 1.0f);
}
\end{lstlisting}
\vspacebig


The next example transposes a 4x4 matrix, using the AVX2 instruction set.

\begin{example}
\label{exampleTranspose4x4}
\end{example}
\begin{lstlisting}[frame=single]
void transpose(float matrix[4][4]) {
    Vec8f row01, row23, col01, col23;
    // load first two rows
    row01.load(&matrix[0][0]);
    // load next two rows
    row23.load(&matrix[2][0]);
    // reorder into columns
    col01 = blend8f<0,4, 8,12,1,5, 9,13>(row01, row23);
    col23 = blend8f<2,6,10,14,3,7,11,15>(row01, row23);
    // store columns into rows
    col01.store(&matrix[0][0]);
    col23.store(&matrix[2][0]);
}
\end{lstlisting}
\vspacesmall

Same example with AVX512:

\begin{example}
\label{exampleTranspose4x4avx512}
\end{example}
\begin{lstlisting}[frame=single]
void transpose(float matrix[4][4]) {
    Vec16f rows, columns;
    // load entire matrix as rows
    rows.load(&matrix[0][0]);
    // reorder into columns
    columns = permute16f<0,4,8,12,1,5,9,13,
        2,6,10,14,3,7,11,15>(rows);
    // store columns into rows
    columns.store(&matrix[0][0]);
}
\end{lstlisting}
\vspacebig

The next example makes a matrix multiplication of two 4x4 matrixes.

\begin{example}
\label{exampleMatrixMul4x4}
\end{example}
\begin{lstlisting}[frame=single]
void matrixmul(float A[4][4], float B[4][4], float M[4][4]){
    // calculates M = A*B
    Vec4f Brow[4], Mrow[4];
    int i, j;
    // load B as rows
    for (i = 0; i < 4; i++) {
        Brow[i].load(&B[i][0]);
    }
    // loop for A and M rows
    for (i = 0; i < 4; i++) {
        Mrow[i] = Vec4f(0.0f);
        // loop for A columns, B rows
        for (j = 0; j < 4; j++) {
            Mrow[i] += Brow[j] * A[i][j];
        }
    }
    // store M
    for (i = 0; i < 4; i++) {
        Mrow[i].store(&M[i][0]);
    }
}
\end{lstlisting}
\vspacebig


The next example makes a table of the sin function and gets sin(x) and cos(x) by table lookup.

\begin{example}
\label{exampleSinTable}
\end{example}
\begin{lstlisting}[frame=single]

#include <cmath>

const double pi = 3.14159265358979323846;

// length of table. Must be a power of 2.
#define sin_tablelen 1024
// the accuracy of table lookup is +/- pi/sin_tablelen

class SinTable {
protected:
    float table[sin_tablelen];
    float resolution;
    float rres;  // 1./resolution
public:
    SinTable();  // constructor
    Vec4f sin(Vec4f x);
    Vec4f cos(Vec4f x);
};

SinTable::SinTable() {  // constructor
    // compute resolution
    resolution = float(2.0 * pi / sin_tablelen);
    rres = 1.0f / resolution;
    // Initialize table (No need to use vectors here because this 
    // is calculated only once:)
    for (int i = 0; i < sin_tablelen; i++) {
        table[i] = sinf((float)i * resolution);
    }
}

Vec4f SinTable::sin(Vec4f x) {
    // calculate sin by table lookup
    Vec4i index = roundi(x * rres);
    // modulo tablelen equivalent to modulo 2*pi
    index &= sin_tablelen - 1;
    // look up in table
    return lookup<sin_tablelen>(index, table);
}

Vec4f SinTable::cos(Vec4f x) {
    // calculate cos by table lookup
    Vec4i index = roundi(x * rres) + sin_tablelen/4;
    // modulo tablelen equivalent to modulo 2*pi
    index &= sin_tablelen - 1;
    // look up in table
    return lookup<sin_tablelen>(index, table);
}

int main() {
    SinTable sintab;
    Vec4f a(0.0f, 0.5f, 1.0f, 1.5f);
    Vec4f b = sintab.sin(a);
    // b = (0.0000 0.4768 0.8416 0.9973)
    // accuracy +/- 0.003
    ...
    return 0;
}
\end{lstlisting}
\vspacesmall


\end{document}