% chapter included in vclmanual.tex
\documentclass[vcl_manual.tex]{subfiles}
\begin{document}

\chapter{Mathematical functions}\label{chap:MathematicalFunctions}
\flushleft

\begin{tabular}{|p{25mm}|p{100mm}|}
\hline
\bfseries Function & exponent \\ \hline
\bfseries Defined for & all floating point vector classes \\ \hline
\bfseries Description & extracts the exponent part of a floating point number. The result is an integer vector.\newline 
exponent(a) = floor(log2(abs(a))).\newline
The value for a = 0 is implementation dependent.\newline 
Subnormal numbers are not supported. \\ \hline
 \bfseries Efficiency & medium \\ \hline
\end{tabular}
\begin{lstlisting}[frame=none]
// Example:
Vec4f a(1.0f, 2.0f, 3.0f, 4.0f);
Vec4i b = exponent(a);  // b = (0, 1, 1, 2)
\end{lstlisting}


\vspacesmall
\begin{tabular}{|p{25mm}|p{100mm}|}
\hline
\bfseries Function & fraction \\ \hline
\bfseries Defined for & all floating point vector classes \\ \hline
\bfseries Description & extracts the fraction part of a floating point number.\newline
a = pow(2, exponent(a)) * fraction(a) \newline
The results for a = 0, subnormal, INF, or NAN are implementation dependent. \\ \hline
\bfseries Efficiency & medium \\ \hline
\end{tabular}
\begin{lstlisting}[frame=none]
// Example:
Vec4f a(2.0f, 3.0f, 4.0f, 5.0f);
Vec4f b = fraction(a);  // b = (1.00f, 1.50f, 1.00f, 1.25f)
\end{lstlisting}


\vspacesmall
\begin{tabular}{|p{25mm}|p{100mm}|}
\hline
\bfseries Function & exp2 \\ \hline
\bfseries Defined for & all floating point vector classes \\ \hline
\bfseries Description & calculates integer powers of 2. The input is an integer vector, the output is a floating point vector. Overflow gives +INF, underflow gives zero. This function will never produce subnormals, and never raise exceptions \\ \hline
\bfseries Efficiency & medium \\ \hline
\end{tabular}
\begin{lstlisting}[frame=none]
// Example:
Vec4i a(-1, 0, 1, 2);
Vec4f b = exp2(a);     // b = (0.5f, 1.0f, 2.0f, 4.0f)
\end{lstlisting}
\vspacebig

\begin{tabular}{|p{25mm}|p{100mm}|}
\hline
\bfseries Function & mul\_add \newline
nmul\_add \newline
mul\_sub \\ \hline
\bfseries Defined for & all floating point vector classes \\ \hline
\bfseries Description & mul\_add(a,b,c) = a*b+c \newline
nmul\_add(a,b,c) = -a*b+c \newline
mul\_sub(a,b,c) = a*b-c \newline
These functions use fused multiply-and-add (FMA) instructions if available. Some compilers use FMA instructions automatically for expressions like a*b+c. Use these functions for optimal performance on all compilers or to specify calculation order, etc. \\ \hline
\bfseries Precision & The intermediate product a*b is calculated with unlimited precision if the FMA instruction set is enabled. \\ \hline
\bfseries Efficiency & good \\ \hline
\end{tabular}
\vspacebig


\begin{tabular}{|p{25mm}|p{100mm}|}
\hline
\bfseries Function & fremainder \newline
fmodulo \\ \hline
\bfseries Defined for & single and double precision floating point vectors \\ \hline
\bfseries Description & 
vector fremainder(vector n, double d) \newline
vector fmodulo(vector n, double d) \newline
n (numerator) is reduced modulo d (denominator). \newline
The same denominator is applied to all vector elements. \newline
The result is within the following limits: \newline
fremainder: -d/2 <= result < d/2 \newline
fmodulo: 0 <= result < d \newline
Note that fmodulo never gives a negative result even if n is negative, unlike the standard fmod function. \\ \hline
\bfseries Precision & d is double precision, even if n is single precision. 
The full double precision of d is utilized. It is recommended to calculate
d with double precision, even if n is single precision. \newline
Precision and efficiency is best if the FMA instruction set is enabled. \\ \hline
\bfseries Efficiency & medium \\ \hline
\end{tabular}
\vspacebig


\section{Floating point categorization functions}\label{FloatingPointCategorizationFunctions}

\begin{tabular}{|p{25mm}|p{100mm}|}
\hline
\bfseries Function & is\_finite \\ \hline
\bfseries Defined for & all floating point vector classes \\ \hline
\bfseries Description & returns a boolean vector with true for elements that are normal, subnormal or zero, false for INF and NAN \\ \hline
\bfseries Efficiency & medium \\ \hline
\end{tabular}
\begin{lstlisting}[frame=none]
// Example:
Vec4f  a( 0.0f, 1.0f, 2.0f, 3.0f);
Vec4f  b(-1.0f, 0.0f, 1.0f, 2.0f);
Vec4f  c = a / b;
Vec4fb d = is_finite(c);  // d = (true, false, true, true)
\end{lstlisting}
\vspacebig


\begin{tabular}{|p{25mm}|p{100mm}|}
\hline
\bfseries Function & is\_inf \\ \hline
\bfseries Defined for & all floating point vector classes \\ \hline
\bfseries Description & returns a boolean vector with true for elements that are +INF or -INF, false for all other values, including NAN \\ \hline
\bfseries Efficiency & good \\ \hline
\end{tabular}
\begin{lstlisting}[frame=none]
// Example:
Vec4f  a( 0.0f, 1.0f, 2.0f, 3.0f);
Vec4f  b(-1.0f, 0.0f, 1.0f, 2.0f);
Vec4f  c = a / b;
Vec4fb d = is_inf(c);  // d = (false, true, false, false)
\end{lstlisting}


\vspacesmall
\begin{tabular}{|p{25mm}|p{100mm}|}
\hline
\bfseries Function & is\_nan \\ \hline
\bfseries Defined for & all floating point vector classes \\ \hline
\bfseries Description & returns a boolean vector with true for all types of NAN, false for all other values, including INF \\ \hline
\bfseries Efficiency & good \\ \hline
\end{tabular}
\begin{lstlisting}[frame=none]
// Example:
Vec4f  a(-1.0f, 0.0f, 1.0f, 2.0f);
Vec4f  b = sqrt(a);
Vec4fb c = is_nan(b);  // c = (true, false, false, false)
\end{lstlisting}


\vspacesmall
\begin{tabular}{|p{25mm}|p{100mm}|}
\hline
\bfseries Function & is\_subnormal \\ \hline
\bfseries Defined for & all floating point vector classes \\ \hline
\bfseries Description & returns a boolean vector with true for subnormal (denormal) vector elements, false for normal numbers, INF and NAN \\ \hline
\bfseries Efficiency & medium \\ \hline
\end{tabular}
\begin{lstlisting}[frame=none]
// Example:
Vec4f  a(1.0f, 1.0E-10f, 1.0E-20f, 1.0E-30f);
Vec4f  b = a * a;           // b = (1.0f, 1.E-20f, 1.E-40f, 0.f)
Vec4fb c = is_subnormal(b); // c = (false,false,true,false)
\end{lstlisting}


\vspacesmall
\begin{tabular}{|p{25mm}|p{100mm}|}
\hline
\bfseries Function & is\_zero\_or\_subnormal \\ \hline
\bfseries Defined for & all floating point vector classes \\ \hline
\bfseries Description & returns a boolean vector with true for zero and subnormal (denormal) vector elements, false for nonzero normal numbers, INF and NAN \\ \hline
\bfseries Efficiency & good \\ \hline
\end{tabular}
\begin{lstlisting}[frame=none]
// Example:
Vec4f  a(1.0f, 1.0E-10f, 1.0E-20f, 1.0E-30f);
Vec4f  b = a * a;           // b = (1.0f, 1.E-20f, 1.E-40f, 0.f)
Vec4fb c = is_zero_or_subnormal(b); // c = (false,false,true,true)
\end{lstlisting}


\vspacesmall
\begin{tabular}{|p{25mm}|p{100mm}|}
\hline
\bfseries Function & 
infinite8h, infinite16h, infinite32h, \newline
infinite4f, infinite8f, infinite16f, \newline
infinite2d, infinite4d, infinite8d \\ \hline
\bfseries Defined for & all floating point vector classes \\ \hline
\bfseries Description & returns positive infinity \\ \hline
\bfseries Efficiency & good \\ \hline
\end{tabular}
\begin{lstlisting}[frame=none]
// Example:
Vec4f  a = infinite4f(); // a = (INF, INF, INF, INF)
\end{lstlisting}
\vspacebig


\label{nan4f}
\begin{tabular}{|p{25mm}|p{100mm}|}
\hline
\bfseries Function & 
nan8h(unsigned int n) \newline
nan16h(unsigned int n) \newline
nan32h(unsigned int n) \newline
nan4f(unsigned int n) \newline
nan8f(unsigned int n) \newline
nan16f(unsigned int n) \newline
nan2d(unsigned int n) \newline
nan4d(unsigned int n) \newline
nan8d(unsigned int n) \\ \hline
\bfseries Defined for & all floating point vector classes \\ \hline
\bfseries Description & returns not-a-number (NAN). \newline
The optional parameter n may be used for error tracing. \newline
The maximum value of n is 0x003FFFFF for single and double precision, and 0x1FF for half precision.\newline
This function generates a quiet NAN in the following way:\newline
Half precision: The value n is OR'ed with 0x200 to set the quiet bit, and inserted as a payload.\newline
Single precision: The value n is OR'ed with 0x400000 to set the quiet bit, and inserted as a payload.\newline
Double precision: The value n is shifted 29 places to the left for the sake of compatibility with single precision. The value is then OR'ed with $1<<51$ to set the quiet bit.\newline
This parameter n (including the quiet bit) can be retrieved later by the function nan\_code (page \pageref{nanCode}). \\ \hline
\bfseries Efficiency & good \\ \hline
\end{tabular}
\begin{lstlisting}[frame=none]
// Example:
Vec4f a = nan4f(); // a = (NAN, NAN, NAN, NAN)
\end{lstlisting}
\vspacesmall


\section{Floating point control word manipulation functions}\label{FPControlWordManipulationFunctions}

MXCSR is a control word that controls floating point exceptions, rounding mode and subnormal numbers for single and double precision floating point numbers. There is one MXCSR for each thread. 

The MXCSR has the following bits:
\begin{tabular}{|p{25mm}|p{100mm}|}
\hline
\bfseries Bit index & \bfseries meaning \\ \hline
0 & Invalid Operation Flag \\ \hline
1 & Denormal (subnormal) Flag \\ \hline
2 & Divide-by-Zero Flag \\ \hline
3 & Overflow Flag \\ \hline
4 & Underflow Flag \\ \hline
5 & Precision Flag \\ \hline
6 & Denormals (subnormals) Are Zeros \\ \hline
7 & Invalid Operation Mask \\ \hline
8 & Denormal (subnormal) Operation Mask \\ \hline
9 & Divide-by-Zero Mask \\ \hline
10 & Overflow Mask \\ \hline
11 & Underflow Mask \\ \hline
12 & Precision Mask \\ \hline
13-14 & Rounding control: \newline
00: round to nearest or even \newline
01: round down towards -infinity \newline
10: round up towards +infinity \newline
11: round towards zero (truncate) \newline
If the rounding mode is temporarily changed then it must be set back to 00 for the vector class library to work correctly. \\ \hline
15 & Flush to Zero \\ \hline
\end{tabular}
\vspacesmall

Please see programming manuals from Intel or AMD for further explanation.


\vspacebig
\begin{tabular}{|p{25mm}|p{100mm}|}
\hline
\bfseries Function & get\_control\_word \\ \hline
\bfseries Description & reads the MXCSR control word \\ \hline
\bfseries Efficiency & medium \\ \hline
\end{tabular}
\begin{lstlisting}[frame=none]
// Example:
int m = get_control_word();  // default value m = 0x1F80
\end{lstlisting}


\vspacesmall
\begin{tabular}{|p{25mm}|p{100mm}|}
\hline
\bfseries Function & set\_control\_word(n) \\ \hline
\bfseries Description & writes the MXCSR control word \\ \hline
\bfseries Efficiency & medium \\ \hline
\end{tabular}
\begin{lstlisting}[frame=none]
// Example:  
// Enable overflow and divide by zero exceptions:
set_control_word(0x1980); 
\end{lstlisting}


\vspacesmall
\begin{tabular}{|p{25mm}|p{100mm}|}
\hline
\bfseries Function & reset\_control\_word \\ \hline
\bfseries Description & sets the MXCSR control word to the default value \\ \hline
\bfseries Efficiency & medium \\ \hline
\end{tabular}
\begin{lstlisting}[frame=none]
// Example:
reset_control_word();
\end{lstlisting}

\label{noSubnormals}
\vspacesmall
\begin{tabular}{|p{25mm}|p{100mm}|}
\hline
\bfseries Function & no\_subnormals \\ \hline
\bfseries Description & Disables the use of subnormal (denormal) values. \newline
Floating point numbers with an absolute value below  \newline
1.18E-38 for single precision or 2.22E-308 for double precision are represented by subnormal numbers. The handling of subnormal numbers is extremely time-consuming on many CPUs. The no\_subnormals function sets the "denormals are zeros" and "flush to zero" mode to avoid the use of subnormal numbers. It is recommended to call this function at the beginning of each thread in order to improve the speed of mathematical calculations if very low numbers are likely to occur. This function has no effect on half precision numbers. \\ \hline
\bfseries Efficiency & medium \\ \hline
\end{tabular}
\begin{lstlisting}[frame=none]
// Example:
no_subnormals();
\end{lstlisting}


\section{Standard mathematical functions}\label{FPMathematicalFunctions}

Standard mathematical functions such as logarithms, exponential functions, power, trigonometric functions, etc. for vectors are available in two versions: as inline code and as an external function library provided by Intel. These functions all take vectors as input and produce vectors as output.
\vspacesmall


The use of vector math functions is straightforward:
\begin{example}
\label{exampleSinFunction}
\end{example} % frame disappears if I put this after end lstlisting
\begin{lstlisting}[frame=single]
#include <stdio.h>
#include "vectorclass.h"
#include "vectormath_trig.h"    // trigonometric functions

int main() {
    Vec4f a(0.0f, 0.5f, 1.0f, 1.5f);// define vector
    Vec4f b = sin(a);               // sine function
    // b = (0.0000f, 0.4794f, 0.8415f, 0.9975f)

    // output results:
    for (int i = 0; i < b.size(); i++) {
        printf("%6.4f ", b[i]);
    }
    printf("\n");
    return 0;
}

\end{lstlisting}
\vspacesmall

The inline versions and the external library versions are using different calculation methods. The inline versions may be faster in some cases, while the external library versions may be faster in other cases. Both versions are many times faster than standard (scalar) math function libraries. 
\vspacesmall

The available vector math functions are listed below. The efficiency is listed as poor because mathematical functions take more time to execute than most other functions, but they are still much faster than scalar alternatives. The details listed apply to the inline version. Details for the library version may be sought in the documentation for the Intel compiler.
\vspacebig


\section{Inline mathematical functions}\label{InlineMathematicalFunctions}

The inline mathematical functions are available by including the appropriate header file, e. g. vectormath\_exp.h for powers, logarithms and exponential functions, and vectormath\_trig.h for trigonometric functions. An advantage of the inline version is that the compiler can optimize the code across function calls, eliminate common sub-expressions, etc. The disadvantage is that you may get multiple instances of the same function taking up space in the code cache.
\vspacesmall

The accuracy is good. The calculation error is typically below 2 ULP (Unit in the Last Place = least significant bit) on the output. (The relative value of one ULP is $2^{-52}$ for double precision and $2^{-23}$ for single precision). Where a function is steep, the maximum error corresponds to 1 ULP at the input. 
Cases where the error can exceed 3 ULP are mentioned under the specific function.
\vspacesmall

The functions do not generate exceptions or set \codei{errno} when an input is out of range. This would be inefficient and it would be problematic for the error handler to detect which vector element caused the error. Instead, the functions return INF (infinity) or NAN (not a number) in case of error. Generally, an overflow will produce INF. A negative overflow produces -INF. An underflow towards zero returns 0. Other errors produce NAN. An efficient way of detecting errors is to let the INF and NAN codes propagate through the calculations and detect the error at the end of a series of calculations as explained on page \pageref{FloatingPointErrors}. It is possible to include an error code in a NAN and detect it with the function nan\_code on page \pageref{nanCode}.
\vspacesmall

Note that many of the inline math functions do not support subnormal numbers. Subnormal numbers may be treated as zero by the logarithm, exponential, power, and root functions. It is recommended to set the “denormals are zero” and “flush to zero” flags by calling the function \codei{no\_subnormals()} first (see page \pageref{noSubnormals}). This may speed up some calculations and give more consistent results.
\vspacesmall

A description of each mathematical function is given below.
\vspacesmall


\section{Using an external library for mathematical functions}\label{ExternalMathLibrary}

A function library made by Intel called SVML (Short Vector Math Library) can be used as an alternative to the inline mathematical functions. SVML is a highly optimized function library that calculates mathematical functions on vectors.
\vspacesmall

The SVML library is part of an Intel compiler installation. The vector class library provides a header file named vectormath\_lib.h that makes it possible to use the Intel SVML library with other compilers. The SVML library is optimized for Intel processors, but it works well with AMD processors as well according to my tests, unless you are using the Intel ICC or ICL compiler (named "classic"). Use the newer Intel ICPX compiler instead, or any other compiler. The SVML library is available for all platforms relevant to the vector class library.
\vspacesmall

\textbf{The SVML library for Windows can be obtained in the following way:}

Install the Intel C++ compiler. You need the files named svml\_dispmt.lib and libircmt.lib. These files can be found in the installation directory, for example:\\
C:\textbackslash Program Files (x86)\textbackslash Intel\textbackslash oneAPI\textbackslash compiler\textbackslash 2022.1.0\textbackslash windows\textbackslash compiler\textbackslash lib\textbackslash intel64\_win
\vspacesmall

Note that there is a 32-bit version and a 64-bit version of each library. We generally prefer to compile vector code for 64-bit mode, so you will probably need the 64-bit versions only. You also need the library file svmlpatch.lib which you can find at the VCL Github site under 
\href{https://github.com/vectorclass/miscellaneous/tree/master/svmlpatch}{miscellaneous}. 
\vspacesmall

svml\_dispmt.lib contains the mathematical vector functions. libircmt.lib contains a function dispatcher used by svml\_dispmt.lib. 
The purpose of svmlpatch.lib is to fix a non-standard calling convention in the SVML library. svmlpatch.lib is only needed in 64-bit mode Windows.
\vspacesmall

Copy the library files svml\_dispmt.lib, svml\_dispmt.lib, and svmlpatch.lib to a suitable location and add them to your C++ project.
\vspacesmall


\textbf{The SVML library for Linux can be obtained in the following way:}

Install the Intel C++ compiler. You need the files named libsvml.a and libirc.a. These files can be found in the installation directory, for example:\\
$\sim$/intel/oneapi/compiler/2022.1.0/linux/compiler/lib/intel64\_lin/
\vspacesmall

Note that there is a 32-bit version and a 64-bit version of each library. We generally prefer to compile vector code for 64-bit mode, so you will probably need the 64-bit version only. 
\vspacesmall

libsvml.a contains the mathematical vector functions, and libircmt.a contains a function dispatcher used by libsvml.a. Copy these two library files to a suitable location and add them to your C++ project.
\vspacesmall


\textbf{Using the library functions in vector code:}

Include the header file vectormath\_lib.h if you want to use the SVML library. Do not include vectormath\_exp.h, vectormath\_trig.h, or vectormath\_hyp.h. 
It is not possible to mix the two kinds of mathematical functions (inline and library) in the same C++ file. The available vector math functions are listed below.
\vspacesmall


\section{Powers, exponential functions and logarithms}\label{ExpLogFunctions}

\vspacesmall
\begin{tabular}{|p{30mm}|p{120mm}|}
\hline
\bfseries Function & pow(vector, vector), pow(vector, scalar) \\ \hline
\bfseries Defined for & single and double precision floating point vectors \\ \hline
\bfseries Inline version & vectormath\_exp.h \\ \hline
\bfseries Library version & vectormath\_lib.h \\ \hline
\bfseries Description & pow(a,b) = $a^b$ \newline 
See also faster alternatives below for integer and rational powers. \\ \hline
\bfseries Range & Subnormal numbers are treated as zero. The result is NAN if a is negative and b is not an integer. NAN's are always propagated by the inline version of pow, even in cases where the IEEE 754 standard specifies otherwise. The library version may fail to propagate NANs in the cases pow(NAN,0) and pow(1,NAN). \\ \hline
\bfseries Precision & better than (0.8*abs(b)+2) ULP \\ \hline
\bfseries Efficiency & poor \\ \hline
\end{tabular}
\begin{lstlisting}[frame=none]
// Example:
Vec4f a( 1.0f,  2.0f, 3.0f, 4.0f);
Vec4f b( 0.0f, -1.0f, 0.5f, 2.0f);
Vec4f c = pow(a, b);
// c = (1.0000, 0.5000, 1.7321, 16.0000)
Vec4f d = pow(a, 2.4f);
// d = (1.0000, 5.2780, 13.9666, 27.8576)
\end{lstlisting}


\vspacesmall
\begin{tabular}{|p{30mm}|p{120mm}|}
\hline
\bfseries Function & pow(vector, int) \\ \hline
\bfseries Defined for & all floating point vector classes \\ \hline
\bfseries Inline version & no extra header file required \\ \hline
\bfseries Library version & not available \\ \hline
\bfseries Description & see page \pageref{powVectorInt} \\ \hline
\bfseries Efficiency & medium \\ \hline
\end{tabular}
\begin{lstlisting}[frame=none]
// Example:
Vec4f a(0.0f, 1.0f, 2.0f, 3.0f);
int   b = 3;
Vec4f c = pow(a, b);  // c = (0.0f, 1.0f, 8.0f, 27.0f)
\end{lstlisting}


\vspacesmall
\begin{tabular}{|p{30mm}|p{120mm}|}
\hline
\bfseries Function & pow\_const(vector, const int) \\ \hline
\bfseries Defined for & all floating point vector classes \\ \hline
\bfseries Inline version & no extra header file required \\ \hline
\bfseries Library version & not available \\ \hline
\bfseries Description & see page \pageref{powConstVectorInt} \\ \hline
\bfseries Efficiency & medium, often better than pow(vector, int) \\ \hline
\end{tabular}
\begin{lstlisting}[frame=none]
// Example:
Vec4f a(0.0f, 1.0f, 2.0f, 3.0f);
Vec4f c = pow_const(a, 3);  // c = (0.0f, 1.0f, 8.0f, 27.0f)
\end{lstlisting}


\vspacesmall
\begin{tabular}{|p{30mm}|p{120mm}|}
\hline
\bfseries Function & pow\_ratio(vector x, const int a, const int b) \\ \hline
\bfseries Defined for & single and double precision floating point vectors \\ \hline
\bfseries Inline version & vectormath\_exp.h \\ \hline
\bfseries Library version & not available \\ \hline
\bfseries Description & Raises all elements of x to the rational power a/b.\newline
a and b must be compile-time constant integers. \\ \hline
\bfseries Range & x may be zero only if a and b are positive. x may be negative only if b is odd.\newline
The range is the same as for cbrt (page \pageref{cbrt}) if b is 3.
 The result when x is infinite may be NAN in some cases.
Subnormal numbers are treated as zero in some cases.  \\ \hline
\bfseries Precision & slightly imprecise for extreme values of \codei{a} due to accumulating rounding errors. 
The precision is similar to the cbrt function when b is 3 or 6. \\ \hline
\bfseries Efficiency & Quite good for b = 1, 2, 4, or 8. Reasonable for b = 3 or 6. No better than pow for other values of b. \\ \hline
\end{tabular}
\begin{lstlisting}[frame=none]
// Example:
Vec4f a(1.0f, 2.0f, 3.0f, 4.0f);
// Reciprocal square root
Vec4f b = pow_ratio(a, -1, 2);  // c = (1.0, 0.707, 0.577, 0.500)
\end{lstlisting}


\vspacesmall
\begin{tabular}{|p{30mm}|p{120mm}|}
\hline
\bfseries Function & exp \\ \hline
\bfseries Defined for & all floating point vector classes \\ \hline
\bfseries Inline version & vectormath\_exp.h \\ \hline
\bfseries Library version & vectormath\_lib.h \\ \hline
\bfseries Description & exponential function $e^x$ \\ \hline
\bfseries Range & double: abs(x) \textless 708.39. float: abs(x) \textless 87.3 \\ \hline
\bfseries Efficiency & Poor. The performance of the inline version for single precision vectors (Vec16f etc.) is better when the instruction set AVX512ER is supported. The performance can be improved further, at a slight loss of precision, when VCL\_FASTEXP is defined in addition to AVX512ER. \\ \hline
\end{tabular}
\begin{lstlisting}[frame=none]
// Example:
#include "vectormath_exp.h"
Vec16f a, b;
b = exp(a);
\end{lstlisting}


\vspacesmall
\begin{tabular}{|p{30mm}|p{120mm}|}
\hline
\bfseries Function & expm1 \\ \hline
\bfseries Defined for & all floating point vector classes \\ \hline
\bfseries Inline version & vectormath\_exp.h \\ \hline
\bfseries Library version & vectormath\_lib.h \\ \hline
\bfseries Description & $e^x-1$. Useful to avoid loss of precision if x is close to 0 \\ \hline
\bfseries Range & double: abs(x) \textless{} 708.39. float: abs(x) \textless{} 87.3 \\ \hline
\bfseries Efficiency & Poor. (not improved with AVX512ER) \\ \hline
\end{tabular}
\vspacebig


\begin{tabular}{|p{30mm}|p{120mm}|}
\hline
\bfseries Function & exp2 \\ \hline
\bfseries Defined for & all floating point vector classes \\ \hline
\bfseries Inline version & vectormath\_exp.h \\ \hline
\bfseries Library version & vectormath\_lib.h \\ \hline
\bfseries Description & $2^x$ \\ \hline
\bfseries Range & double: abs(x) \textless{} 1020. float: abs(x) \textless{} 27. \\ \hline
\bfseries Efficiency & The performance of the inline version is good for single precision vectors if instruction set AVX512ER is supported. (VCL\_FASTEXP is not needed). \newline
Use pow or pow\_const instead if x is an integer. \\ \hline
\end{tabular}
\vspacebig


\begin{tabular}{|p{30mm}|p{120mm}|}
\hline
\bfseries Function & exp10 \\ \hline
\bfseries Defined for & all floating point vector classes \\ \hline
\bfseries Inline version & vectormath\_exp.h \\ \hline
\bfseries Library version & vectormath\_lib.h \\ \hline
\bfseries Description & $10^x$ \\ \hline
\bfseries Range & double: abs(x) \textless{} 307.65. float: abs(x) \textless{} 37.9. \\ \hline
\bfseries Efficiency & Poor. The performance of the inline version for single precision vectors (Vec16f etc.) is better when the instruction set AVX512ER is supported. The performance can be improved further, at a slight loss of precision, when VCL\_FASTEXP is defined in addition to AVX512ER. \newline
Use pow or pow\_const instead if x is an integer. \\ \hline
\end{tabular}
\begin{lstlisting}[frame=none]
// Example:
#include "vectormath_exp.h"
Vec16f a, b;
b = exp10(a); 
\end{lstlisting}


\vspacesmall
\begin{tabular}{|p{30mm}|p{120mm}|}
\hline
\bfseries Function & log \\ \hline
\bfseries Defined for & single and double precision floating point vectors \\ \hline
\bfseries Inline version & vectormath\_exp.h \\ \hline
\bfseries Library version & vectormath\_lib.h \\ \hline
\bfseries Description & natural logarithm \\ \hline
\bfseries Range & The input must be a normal number. Subnormal numbers are treated as zero. \\ \hline
\bfseries Efficiency & poor \\ \hline
\end{tabular}
\vspacebig


\begin{tabular}{|p{30mm}|p{120mm}|}
\hline
\bfseries Function & log1p \\ \hline
\bfseries Defined for & single and double precision floating point vectors \\ \hline
\bfseries Inline version & vectormath\_exp.h \\ \hline
\bfseries Library version & vectormath\_lib.h \\ \hline
\bfseries Description & log(1+x) \newline
Useful to avoid loss of precision if x is close to 0 \\ \hline
\bfseries Range & x \textgreater{} -1 \\ \hline
\bfseries Efficiency & poor \\ \hline
\end{tabular}
\vspacebig


\begin{tabular}{|p{30mm}|p{120mm}|}
\hline
\bfseries Function & log2 \\ \hline
\bfseries Defined for & single and double precision floating point vectors \\ \hline
\bfseries Inline version & vectormath\_exp.h \\ \hline
\bfseries Library version & vectormath\_lib.h \\ \hline
\bfseries Description & logarithm base 2  \\ \hline
\bfseries Range & The input must be a normal number. Subnormal numbers are treated as zero. \\ \hline
\bfseries Efficiency & poor \\ \hline
\end{tabular}
\vspacebig


\begin{tabular}{|p{30mm}|p{120mm}|}
\hline
\bfseries Function & log10 \\ \hline
\bfseries Defined for & single and double precision floating point vectors \\ \hline
\bfseries Inline version & vectormath\_exp.h \\ \hline
\bfseries Library version & vectormath\_lib.h \\ \hline
\bfseries Description & logarithm base 10  \\ \hline
\bfseries Range & The input must be a normal number. Subnormal numbers are treated as zero. \\ \hline
\bfseries Efficiency & poor \\ \hline
\end{tabular}
\vspacebig

\label{cbrt}
\begin{tabular}{|p{30mm}|p{120mm}|}
\hline
\bfseries Function & cbrt \\ \hline
\bfseries Defined for & single and double precision floating point vectors \\ \hline
\bfseries Inline version & vectormath\_exp.h \\ \hline
\bfseries Library version & vectormath\_lib.h \\ \hline
\bfseries Description & cube root  \\ \hline
\bfseries Range & float: 0, $\pm 10^{-28} ..{}  10^{28}$ \newline
                  double: 0, $\pm 10^{-200} ..{}  10^{200}$ \newline
                  The return value is 0 if abs(x) is too small \\ \hline
\bfseries Precision & 5 ULP \\ \hline                  
\bfseries Efficiency & Faster than pow \\ \hline
\end{tabular}

\vspacebig


\section{Trigonometric functions and inverse trigonometric functions}
The functions sin, cos, sincos, and tan take arguments in radians. The functions sinpi, cospi, sincospi, and tanpi take arguments in multiples of \pi{}. The inverse trigonometric functions return arguments in radians.
\vspacesmall

The trigonometric functions with arguments in radians have decreasing precision for extreme values of the argument. For example, sin(1000000*pi) gives the result -2.2E-10, while sinpi(1000000) gives the exact result 0. The reasons for the inaccuracy are that the value of \pi{} cannot be represented exactly, and that the reduction modulo \pi{} inside the sin function involves loss of precision for extreme values of the argument. It is recommended to use the sinpi, cospi, sincospi, and tanpi functions when accuracy is important and the algorithm allows using \pi{} as a unit rather than radians.
\vspacebig

\begin{tabular}{|p{30mm}|p{120mm}|}
\hline
\bfseries Function & sin \\ \hline
\bfseries Defined for & all floating point vector classes \\ \hline
\bfseries Inline version & vectormath\_trig.h \\ \hline
\bfseries Library version & vectormath\_lib.h \\ \hline
\bfseries Description & sine function \\ \hline
\bfseries Range & abs(x) is limited to 314, 1.E7, 1.E15 for half, single, and double precision, respectively. This limit is lower if FMA instructions are not supported. The result is 0 for big x. The result may be 0 or NAN when the input is infinity depending on the implementation. \\ \hline
\bfseries Efficiency & poor \\ \hline
\end{tabular}
\begin{lstlisting}[frame=none]
// Example:
Vec4f a(0.0f, 0.5f, 1.0f, 1.5f);// define vector
Vec4f b = sin(a);               // sine function
// b = (0.0000f, 0.4794f, 0.8415f, 0.9975f)
\end{lstlisting}


\vspacesmall
\begin{tabular}{|p{30mm}|p{120mm}|}
\hline
\bfseries Function & cos \\ \hline
\bfseries Defined for & all floating point vector classes \\ \hline
\bfseries Inline version & vectormath\_trig.h \\ \hline
\bfseries Library version & vectormath\_lib.h \\ \hline
\bfseries Description & cosine function \\ \hline
\bfseries Range & abs(x) is limited to 314, 1.E7, 1.E15 for half, single, and double precision, respectively. This limit is lower if FMA instructions are not supported. The result is 1 for big x. The result may be 1 or NAN when the input is infinity depending on the implementation \\ \hline
\bfseries Efficiency & poor \\ \hline
\end{tabular}
\vspacebig


\begin{tabular}{|p{30mm}|p{120mm}|}
\hline
\bfseries Function & sincos \\ \hline
\bfseries Defined for & all floating point vector classes \\ \hline
\bfseries Inline version & vectormath\_trig.h \\ \hline
\bfseries Library version & vectormath\_lib.h (not with MS compiler) \\ \hline
\bfseries Description & sine and cosine computed simultaneously.\\ \hline
\bfseries Range & abs(x) is limited to 314, 1.E7, 1.E15 for half, single, and double precision, respectively. This limit is lower if FMA instructions are not supported. The result is 0 and 1 for big x. The result may or may not be NAN when the input is infinity. \\ \hline
\bfseries Efficiency & faster than computing sin and cos separately \\ \hline
\end{tabular}
\begin{lstlisting}[frame=none]
// Example:
Vec4f a(0.0f, 0.5f, 1.0f, 1.5f);
Vec4f s, c;
s = sincos(&c, a);
// s = (0.0000, 0.4794, 0.8415, 0.9975)
// c = (1.0000, 0.8776, 0.5403, 0.0707)
\end{lstlisting}


\vspacesmall
\begin{tabular}{|p{30mm}|p{120mm}|}
\hline
\bfseries Function & tan \\ \hline
\bfseries Defined for & all floating point vector classes \\ \hline
\bfseries Inline version & vectormath\_trig.h \\ \hline
\bfseries Library version & vectormath\_lib.h \\ \hline
\bfseries Description & tangent function \\ \hline
\bfseries Range & abs(x) is limited to 314, 1.E7, 1.E15 for half, single, and double precision, respectively. This limit is lower if FMA instructions are not supported. The result is 0 for big x. The result may be 0 or NAN when the input is infinity depending on the implementation. \\ \hline
\bfseries Efficiency & poor \\ \hline
\end{tabular}
\vspacebig

\vspacesmall
\begin{tabular}{|p{30mm}|p{120mm}|}
\hline
\bfseries Function & sinpi, cospi, sincospi, tanpi \\ \hline
\bfseries Defined for & all floating point vector classes \\ \hline
\bfseries Inline version & vectormath\_trig.h \\ \hline
\bfseries Library version & vectormath\_lib.h. Not with Intel compiler. sincospi not available \\ \hline
\bfseries Description & sinpi(x) = sin(pi*x), etc. \newline
The ...pi functions are more accurate than the normal trigonometric functions when x is a multiple or simple fraction of $\pi$ or x is high. For example, tanpi(0.5) gives INF while tan(pi*0.5) gives a high number less than INF because $\pi/2$ cannot be represented exactly. tanpi(n+0.5) gives INF for n even, and -INF for n odd, in accordance with the IEEE754-2019 standard. The standard for signed zero results is not necessarily followed. \\ \hline
\bfseries Range & numerically high values of x are interpreted as even integers, giving exact results. The result may or may not be NAN when the input is infinity. \\ \hline
\bfseries Efficiency & same as normal trigonometric functions, or slightly better \\ \hline
\end{tabular}
\vspacebig


\begin{tabular}{|p{30mm}|p{120mm}|}
\hline
\bfseries Function & asin \\ \hline
\bfseries Defined for & single and double precision floating point vectors \\ \hline
\bfseries Inline version & vectormath\_trig.h \\ \hline
\bfseries Library version & vectormath\_lib.h \\ \hline
\bfseries Description & inverse sine function \\ \hline
\bfseries Range & -1 $\leq$ x $\leq$ 1 \\ \hline
\bfseries Efficiency & poor \\ \hline
\end{tabular}
\vspacebig


\begin{tabular}{|p{30mm}|p{120mm}|}
\hline
\bfseries Function & acos \\ \hline
\bfseries Defined for & single and double precision floating point vectors \\ \hline
\bfseries Inline version & vectormath\_trig.h \\ \hline
\bfseries Library version & vectormath\_lib.h \\ \hline
\bfseries Description & inverse cosine function \\ \hline
\bfseries Range & -1 $\leq$ x $\leq$ 1 \\ \hline
\bfseries Efficiency & poor \\ \hline
\end{tabular}
\vspacebig


\begin{tabular}{|p{30mm}|p{120mm}|}
\hline
\bfseries Function & atan \\ \hline
\bfseries Defined for & single and double precision floating point vectors \\ \hline
\bfseries Inline version & vectormath\_trig.h \\ \hline
\bfseries Library version & vectormath\_lib.h \\ \hline
\bfseries Description & Inverse tangent \\ \hline
\bfseries Range & Results between $-\pi/2$ and $\pi/2$ \\ \hline
\bfseries Efficiency & poor \\ \hline
\end{tabular}
\vspacebig


\begin{tabular}{|p{30mm}|p{120mm}|}
\hline
\bfseries Function & atan2 \\ \hline
\bfseries Defined for & single and double precision floating point vectors \\ \hline
\bfseries Inline version & vectormath\_trig.h \\ \hline
\bfseries Library version & vectormath\_lib.h \\ \hline
\bfseries Description & Inverse tangent with two parameters, x and y, gives the angle to a point in the (x,y) plane \\ \hline
\bfseries Range & Results between $-\pi$ and $\pi$ \newline
The result of atan2(0,0) is 0 by convention\\ \hline
\bfseries Efficiency & poor \\ \hline
\end{tabular}
\vspacesmall


\section{Hyperbolic functions and inverse hyperbolic functions}\label{HyperbolicFunctions}

\vspacesmall
\begin{tabular}{|p{30mm}|p{120mm}|}
\hline
\bfseries Function & sinh \\ \hline
\bfseries Defined for & single and double precision floating point vectors \\ \hline
\bfseries Inline version & vectormath\_hyp.h \\ \hline
\bfseries Library version & vectormath\_lib.h \\ \hline
\bfseries Description & hyperbolic sine \\ \hline
\bfseries Range & double: abs(x) \textless{} 709. float: abs(x) \textless{} 88. \\ \hline
\bfseries Efficiency & poor \\ \hline
\end{tabular}
\vspacebig


\begin{tabular}{|p{30mm}|p{120mm}|}
\hline
\bfseries Function & cosh \\ \hline
\bfseries Defined for & single and double precision floating point vectors \\ \hline
\bfseries Inline version & vectormath\_hyp.h \\ \hline
\bfseries Library version & vectormath\_lib.h \\ \hline
\bfseries Description & hyperbolic cosine \\ \hline
\bfseries Range & double: abs(x) \textless{} 709. float: abs(x) \textless{} 88. \\ \hline
\bfseries Efficiency & poor \\ \hline
\end{tabular}
\vspacebig


\begin{tabular}{|p{30mm}|p{120mm}|}
\hline
\bfseries Function & tanh \\ \hline
\bfseries Defined for & single and double precision floating point vectors \\ \hline
\bfseries Inline version & vectormath\_hyp.h \\ \hline
\bfseries Library version & vectormath\_lib.h \\ \hline
\bfseries Description & hyperbolic tangent \\ \hline
\bfseries Efficiency & poor \\ \hline
\end{tabular}
\vspacebig


\begin{tabular}{|p{30mm}|p{120mm}|}
\hline
\bfseries Function & asinh \\ \hline
\bfseries Defined for & single and double precision floating point vectors \\ \hline
\bfseries Inline version & vectormath\_hyp.h \\ \hline
\bfseries Library version & vectormath\_lib.h \\ \hline
\bfseries Description & inverse hyperbolic sine \\ \hline
\bfseries Efficiency & poor \\ \hline
\end{tabular}
\vspacebig


\begin{tabular}{|p{30mm}|p{120mm}|}
\hline
\bfseries Function & acosh \\ \hline
\bfseries Defined for & single and double precision floating point vectors \\ \hline
\bfseries Inline version & vectormath\_hyp.h \\ \hline
\bfseries Library version & vectormath\_lib.h \\ \hline
\bfseries Description & inverse hyperbolic cosine \\ \hline
\bfseries Efficiency & poor \\ \hline
\end{tabular}
\vspacebig


\begin{tabular}{|p{30mm}|p{120mm}|}
\hline
\bfseries Function & atanh \\ \hline
\bfseries Defined for & single and double precision floating point vectors \\ \hline
\bfseries Inline version & vectormath\_hyp.h \\ \hline
\bfseries Library version & vectormath\_lib.h \\ \hline
\bfseries Description & inverse hyperbolic tangent \\ \hline
\bfseries Efficiency & poor \\ \hline
\end{tabular}
\vspacesmall


\section{Other mathematical functions}\label{OtherMathematicalFunctions}

\vspacesmall
\begin{tabular}{|p{30mm}|p{120mm}|}
\hline
\bfseries Function & erf \\ \hline
\bfseries Defined for & single and double precision floating point vectors \\ \hline
\bfseries Inline version & not available \\ \hline
\bfseries Library version & vectormath\_lib.h \\ \hline
\bfseries Description & error function \\ \hline
\bfseries Efficiency & poor \\ \hline
\end{tabular}
\vspacebig


\begin{tabular}{|p{30mm}|p{120mm}|}
\hline
\bfseries Function & erfc \\ \hline
\bfseries Defined for & single and double precision floating point vectors \\ \hline
\bfseries Inline version & not available \\ \hline
\bfseries Library version & vectormath\_lib.h \\ \hline
\bfseries Description & error function complement \\ \hline
\bfseries Efficiency & poor \\ \hline
\end{tabular}
\vspacebig


\begin{tabular}{|p{30mm}|p{120mm}|}
\hline
\bfseries Function & erfinv \\ \hline
\bfseries Defined for & single and double precision floating point vectors \\ \hline
\bfseries Inline version & not available \\ \hline
\bfseries Library version & vectormath\_lib.h \\ \hline
\bfseries Description & inverse error function \\ \hline
\bfseries Efficiency & poor \\ \hline
\end{tabular}
\vspacebig


\begin{tabular}{|p{30mm}|p{120mm}|}
\hline
\bfseries Function & cdfnorm \\ \hline
\bfseries Defined for & single and double precision floating point vectors \\ \hline
\bfseries Inline version & not available \\ \hline
\bfseries Library version & vectormath\_lib.h \\ \hline
\bfseries Description & cumulative normal distribution function \\ \hline
\bfseries Efficiency & poor \\ \hline
\end{tabular}
\vspacebig


\begin{tabular}{|p{30mm}|p{120mm}|}
\hline
\bfseries Function & cdfnorminv \\ \hline
\bfseries Defined for & single and double precision floating point vectors \\ \hline
\bfseries Inline version & not available \\ \hline
\bfseries Library version & vectormath\_lib.h \\ \hline
\bfseries Description & inverse cumulative normal distribution function \\ \hline
\bfseries Efficiency & poor \\ \hline
\end{tabular}
\vspacebig

\label{nanCode}
\begin{tabular}{|p{30mm}|p{120mm}|}
\hline
\bfseries Function & 
Vec8us nan\_code(Vec8h)\newline
Vec16us nan\_code(Vec16h)\newline
Vec32us nan\_code(Vec32h)\newline
Vec4ui nan\_code(Vec4f)\newline
Vec8ui nan\_code(Vec8f)\newline
Vec16ui nan\_code(Vec16f)\newline
Vec2uq nan\_code(Vec2d)\newline
Vec4uq nan\_code(Vec4d)\newline
Vec8uq nan\_code(Vec8d) \\ \hline
\bfseries Defined for & all floating point vector classes \\ \hline
\bfseries Inline version & vectormath\_exp.h \\ \hline
\bfseries Library version & not available \\ \hline
\bfseries Description & Extracts an error code hidden as payload in a NAN. This code can be generated with the functions nan4f etc. (page \pageref{nan4f}) and propagated through a series of calculations. When two NANs are combined (e.g. NAN1+NAN2), current processors propagate the first one. NANs produced by CPU instructions, such as 0./0. or sqrt(-1.) have a code of zero. NANs cannot propagate through integers and booleans.\newline
The return value is the payload including the quiet bit. For double precision, the value is shifted 29 places to the right for the sake of compatibility with single precision.\newline
The sign bit is ignored.\newline
The return value is 0 for inputs that are not NAN. \\ \hline
\bfseries Efficiency & medium \\ \hline
\end{tabular}
\vspacesmall


\end{document}